\documentclass[12pt]{article}
\usepackage[spanish,es-tabla]{babel}
\usepackage[utf8]{inputenc}
\usepackage{graphicx}
\usepackage{tabularx}
\usepackage{hyperref}
\usepackage{fancyhdr}

\begin{document}

%% Título principal
{\LARGE\textbf{Plan de Gestión de las Comunicaciones}}\\[1cm]

\begin{tabular}{ll}
\textbf{Nombre del proyecto:} & {{nombreProyecto}} \\[0.3cm]
\textbf{Identificador del proyecto:} & {{idProyecto}} \\[0.3cm]
\textbf{Fecha elaboración:} & {{fechaElaboracion}} \\
\end{tabular}

\tableofcontents
\newpage

\section{Información del Proyecto}
\begin{tabular}{|p{5cm}|p{8cm}|}
\hline
\textbf{Empresa/Organización} & {{empresaNombre}} \\
\hline
\textbf{Nombre del proyecto} & {{nombreProyecto}} \\
\hline
\textbf{Fecha de elaboración} & {{fechaElaboracion}} \\
\hline
\textbf{Cliente} & {{clienteNombre}} \\
\hline
\textbf{Patrocinador principal} & {{patrocinador}} \\
\hline
\textbf{Director del proyecto} & {{director}} \\
\hline
\end{tabular}

\section{Diagrama de comunicación}
[Aquí se debe insertar el diagrama de comunicación del proyecto]

\section{Matriz de escalamiento}
[Aquí se debe insertar la matriz de escalamiento]

\section{Logística del trabajo}
[Aquí se debe describir la logística del trabajo]

\section{Listado de reportes}
[Aquí se debe incluir el listado de reportes]

\section{Diagrama de flujo de la información}
[Aquí se debe insertar el diagrama de flujo de la información]

\section{Glosario de términos}
[Aquí se debe incluir el glosario de términos]

\end{document}
