\documentclass{article}
\usepackage[table]{xcolor}
\usepackage{tabularx}
\usepackage{geometry}
\geometry{margin=1.5cm}

\newcolumntype{Y}{>{\raggedright\arraybackslash}X}
\newcolumntype{L}{>{\raggedright\arraybackslash\hsize=0.8\hsize}X}
\newcolumntype{R}{>{\raggedright\arraybackslash\hsize=1.2\hsize}X}

\begin{document}

    \begin{center}
    {\huge\textbf{Plan de Gestión de Requisitos}}
    \end{center}

    \vfill

    \renewcommand{\arraystretch}{1.5}

    \noindent
    \begin{tabularx}{0.6\textwidth}{|>{\columncolor{gray!40}}L|>{\columncolor{white}}R|}
        \hline
        \textbf{NOMBRE DEL PROYECTO:} & {{proyectName}} \\
        \hline
        \textbf{ID DEL PROYECTO:} & {{idProyect}} \\
        \hline
        \textbf{FECHA DEL PROYECTO:} & {{elaborationDate}} \\
        \hline
    \end{tabularx}


    \newpage
    
    \tableofcontents
    \newpage
    
    \section{Propósito y Justificación del Proyecto}
    {{purposeJustificationProject}}
    
    \section{Descripción del Alcance}
    {{scopeDescription}}
    
    \section{Requisitos de Alto Nivel}
    {{highLevelRequirements}}
    
    \section{Límites}
    {{boundaries}}
    
    \section{Estrategia}
    {{strategy}}
    
    \section{Entregables}
    {{deliverables}}

    \section{Criterios de Aceptación}
    {{acceptanceCriteria}}

    \section{Restricciones}
    {{restrictions}}

    \section{Estimación de Costos}
    {{costEstimation}}

    \section{Análisis Costo-Beneficio}
    {{costBenefitAnalysis}}

    \vspace{2.5cm}

    \begin{tabular}{@{}p{.5in}p{4in}@{}}
    Aprobado: & \hrulefill \\
    \rowcolor{white} & {{proyectPromotor}} \\
    \rowcolor{white} & {{proyectPromotorTitle}}\\
    \end{tabular}

    
\end{document}