\documentclass{article}
\usepackage[table]{xcolor} % Para colorear celdas
\usepackage{tabularx} % Para tablas ajustables al ancho
\usepackage{geometry} % Para ajustar márgenes
\geometry{margin=1.5cm} % Márgenes pequeños

% Definir nuevos tipos de columnas
\newcolumntype{Y}{>{\raggedright\arraybackslash}X}
\newcolumntype{L}{>{\raggedright\arraybackslash\hsize=0.8\hsize}X}
\newcolumntype{R}{>{\raggedright\arraybackslash\hsize=1.2\hsize}X}

\begin{document}

    \begin{center}
    {\huge\textbf{Plan de Gestión de Requisitos}}\\[1cm]
    \end{center}

    \renewcommand{\arraystretch}{1.5}

    \begin{tabular}{|ll|} \hline
    \rowcolor{gray!10}\textbf{Nombre del proyecto:} & {{proyectName}} \\[0.3cm]
    \textbf{Identificador del proyecto:} & {{idProyect}} \\[0.3cm]
    \textbf{Lider del proyecto:} & {{proyectLeader}} \\[0.3cm]
    \textbf{Lider de QA:} & {{qaLeader}} \\[0.3cm]
    \textbf{Fecha elaboración:} & {{elaborationDate}} \\ \hline
    \end{tabular}

    \newpage
    
    \tableofcontents
    \newpage
    
    \section{Enfoque de Gestión de Requisitos}
    \subsection{Identificación de Requisitos}
    {{requirementsIdentification}}
    \subsection{Análisis de Requisitos}
    {{requirementsAnalysis}}
    \subsection{Documentación de Requisitos}
    {{requirementsDocumentation}}
    \subsection{Gestión Continua de Requisitos}
    {{continuousRequirements}}
    
    \section{Gestión de la Configuración}
    \subsection{Control de cambios}
    {{changeControl}}
    
    \section{Proceso de Priorización de Requisitos}
    {{requirementsPrioritization}}
    
    \section{Métricas del Producto}
    \subsection{Costo}
    {{cost}}
    \subsection{Calidad}
    {{quality}}
    \subsection{Rendimiento}
    {{performance}}

\end{document}